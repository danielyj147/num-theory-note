\documentclass{report}
\usepackage{enumitem}
\usepackage{amsmath} % For expanded math capabilities
\input{preamble}
\input{macros}
\input{letterfonts}

\title{\Huge{Number Theory \& Math Reasoning} \\ Instructor: Dr. Daniel Saracino}
\author{\huge{Note Taken by Daniel Jeong}}
\date{from 2024-08-29 to 2024-12-20}

\begin{document}

\maketitle
\newpage
\pagebreak

\thm{Theorem 1}{For all integers $a$, $b$, $c$, $d$: 
\begin{enumerate}[label=(\roman*)]
  \item If $a \mid b$ and $b \mid c$, then $a \mid c$.
  \item If $a \mid b$ and $a \mid c$, then for all $x, y \in \mathbb{Z}$, $a \mid (xb + yc)$.
  \item If $a \mid b$ and $c \mid d$, then $ac \mid bd$.
\end{enumerate}
}

\thm{Theorem 2}{Suppose $a$ and $b$ are integers that are not both $0$, so that $(a, b)$ exists.  
Then for every $n \in \mathbb{Z}$,
\[ 
(a, b) = (a + nb, b) = (a, b + na)
\]
}

\thm{Theorem 3}{Every integer $n \geq 2$ is the product of one or more primes.}

\thm{Theorem 4}{There exist infinitely many primes.}

\thm{Theorem 5}{For every integer $n \geq 2$, the factorization of $n$ into primes is unique, 
except for the order in which the factors are written.}

\thm{Theorem 6}{Suppose $m \in \mathbb{Z}^+$ and $a, b \in \mathbb{Z}$.
Then the following are equivalent:
\begin{enumerate}[label=(\roman*)]
    \item $a \equiv b \pmod{m}$; that is, $m \mid (a - b)$.
    \item $a = b + m\ell$ for some $\ell \in \mathbb{Z}$.
    \item $\overline{a}^m = \overline{b}^m$.
\end{enumerate}
Thus, the remainders are equal, and any one of (i), (ii), or (iii) proves the other two.
}

\thm{Theorem 7}{Basic Properties of Congruence Modulo $m$:\\
Suppose $m \in \mathbb{Z}^+$, then the following all hold:
\begin{enumerate}[label=(\roman*)]
    \item **Reflexivity**: For every $a \in \mathbb{Z}$, $a \equiv a \pmod{m}$.
    \item **Symmetry**: For all $a, b \in \mathbb{Z}$, if $a \equiv b \pmod{m}$, then $b \equiv a \pmod{m}$.
    \item **Transitivity**: For all $a, b, c \in \mathbb{Z}$, if $a \equiv b \pmod{m}$ and $b \equiv c \pmod{m}$, then $a \equiv c \pmod{m}$.
\end{enumerate}
When a relationship between elements of a set $S$ is reflexive, symmetric, and transitive, the relationship is called an equivalence relation on $S$.\\
Thus, **Theorem 7** says that "congruence modulo $m$" is an equivalence relation on the set $\mathbb{Z}$.
}

\thm{Theorem 8}{Suppose $m \in \mathbb{Z}^+$, and let $a, b, c, d \in \mathbb{Z}$.\\
Then, if $a \equiv b \pmod{m}$ and $c \equiv d \pmod{m}$, we have:
\begin{enumerate}
    \item $a + c \equiv b + d \pmod{m}$.
    \item $ac \equiv bd \pmod{m}$.
\end{enumerate}
}

\mlenma{Bezout's Lemma}{If $m$ and $n$ are integers that are not both zero, 
then there exist $x, y \in \mathbb{Z}$ such that $xm + yn = (m, n)$.}

\mlenma{Euclid's Lemma}{Suppose $a$, $b$, $c$ are integers such that $a \mid bc$ and $(a, b) = 1$.  
Then $a \mid c$.}

\dfn{Coprime}{We say integers $a$ and $b$ are relatively prime (or "coprime") if $(a, b) = 1$.}

\dfn{Prime}{An integer $p$ is prime if $p > 1$ and the only positive integers that divide $p$ are $1$ and $p$.}

\cor{Corollary 1}{Every integer $n \ge 2$ has a unique representation in the form:
\[
n = p_{1}^{e_{1}} \times p_{2}^{e_{2}} \times p_{3}^{e_{3}} \times \cdots \times p_{k}^{e_{k}}
\]
where the $p$'s are distinct primes and each $e_i \ge 1$. This is called the prime-power decomposition of $n$.}

\cor{Corollary 2}{Suppose $n \geq 2$ and the prime-power decomposition (ppd) of $n$ is:
\[
n = p_{1}^{e_{1}} \times p_{2}^{e_{2}} \times p_{3}^{e_{3}} \times \cdots \times p_{k}^{e_{k}}.
\]
Then if $m$ is any positive integer that divides $n$,
\[
m = p_{1}^{f_{1}} \times p_{2}^{f_{2}} \times p_{3}^{f_{3}} \times \cdots \times p_{k}^{f_{k}}
\]
for some integers $f_1, \cdots, f_k$ such that $0 \leq f_i \leq e_i$.}

\cor{Corollary 3}{If
\[
n = p_{1}^{c_{1}} \times p_{2}^{c_{2}} \times p_{3}^{c_{3}} \times \cdots \times p_{k}^{c_{k}}
\]
and
\[
m = p_{1}^{d_{1}} \times p_{2}^{d_{2}} \times p_{3}^{d_{3}} \times \cdots \times p_{k}^{d_{k}},
\]
where $p_1, \cdots, p_k$ are distinct primes, and the $c_i$'s and $d_i$'s are nonnegative integers, then:
\[
(m, n) = p_1^{\min(c_1, d_1)} \times p_2^{\min(c_2, d_2)} \times \cdots \times p_k^{\min(c_k, d_k)}.
\]}

\cor{Corollary 4}{If $n$ is an integer such that $n \geq 2$, then there exists a positive integer $m$ such that $n = m^2$ if and only if all the exponents in the ppd of $n$ are even.}

\cor{Corollary 5: Theorem 8-1}{Suppose $a \equiv b \pmod{m}$. Then:
\begin{align*}
a &\equiv b \pmod{m}, \\
a^2 &\equiv b^2 \pmod{m}, \\
a^3 &\equiv b^3 \pmod{m}, \\
\vdots & \\
a^k &\equiv b^k \pmod{m} \quad \text{for every} \ k \in \mathbb{Z}^+.
\end{align*}

Thus, using part (ii) of Theorem 8 repeatedly, we get:  
If $a \equiv b \pmod{m}$, then $a^k \equiv b^k \pmod{m}$ for every $k \in \mathbb{Z}^+$.}

\cor{Corollary 6: Theorem 8-2}{If $p(x)$ is a polynomial with integer coefficients and $a \equiv b \pmod{m}$,  
then $p(a) \equiv p(b) \pmod{m}$.}

\end{document}
